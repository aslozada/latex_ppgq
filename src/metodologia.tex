%metologia
\newpage 
 \section{Metodologia}
  \subsection{El programa}
   Suponemos que el lector tiene conocimiento prévio de tremodinámica.
   Nuestra intención aquí es revisar brevemente la parte esencial de la
   teoria, para introducir a fenómenos críticos y, al final del capítulo
   poner la teoria en una forma más generalizada que en el inicio.\\

   Así, para tal cambio,..., los pares de variables $(S^{~},T^{~})$, ...,
   ocurren,... de esto sigue que
   \begin{equation}
     T^{~}=\left(\frac{\partial U}{\partial S^{~}}\right)
   \end{equation}

   \subsection{el código}
    Otra funcion de respuesta importante para sistemas magnéticos es la 
    \textit{suceptibilidad magnética isotérmica}$\chi_{T}$ que está definida
    por unidad de volumen de materia.
    \begin{equation}
      \chi_{T}=\left(\frac{\partial \mathcal{M}}{\partial \mathcal{H}}\right)
    \end{equation}
    \subsection{Equilibrio de fase}
    Si las curvas de temperatura constante son trazadas en el plano, se encunetra que, ..., mostrado en la figura~\ref{fig:pollitos}
    \begin{figure}[!h]
      \centering
      \includegraphics[scale=0.5]{frango}
      \caption{esto otro pollito}
      \label{fig:pollitos}
    \end{figure}  
    \subsubsection{el segmento}
    Retornando al sistema general, la densidad de energia libre $f_{\eta}$ definida por la ecuación. Como es mostrado en la tabla~\ref{tab:b}
    \begin{table}[!htb]
        \begin{tabular}{lrc}\hline
          name & mark & grade \\
          \hline
          bla & bla & bla \\
          bla & bla & bla \\
          bla & bla & bla \\ \hline
        \end{tabular}
        \caption{esto es otra tabla}\label{tab:b}
    \end{table}
    \newpage

