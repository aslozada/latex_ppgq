% Master document %--Thesis
%--Author: Lozada-Blanco Asdrubal
%--the following is cc by-sa 3.0 with attribution required
% any request to e-mail: aslozada@gmail.com 
%---------------------------------------------------------------------------------------------------------
%--UNOFFICIAL template to master's and doctoral thesis in concordance with instruction from PPGQ-UFSCar
% https://www.ppgq.ufscar.br/regimentos-normas/normas-redacao-dissertacao-e-tese
%----------------------------------------------------------------------------------------------------------
\documentclass[a4paper,12pt]{article}
%--track & revision  
%------------------
%\usepackage[superscript,biblabel,nomove]{cite}
\usepackage[superscript]{cite}
%%\usepackage[numbers]{natbib}
%------------------
\usepackage{changes}
% replace the option name with the surname of author and supervisor
\definechangesauthor[name={surname-author}, color=blue]{acronymus-author}
\definechangesauthor[name={surname-supervisor}, color=orange]{acronymus-supervisor}
\usepackage{todonotes}
\makeatletter
%  \setremarkmarkup{\todo[color=Changes@Color#1!20,size=\scriptsize]{#1: #2}} 
\makeatother  
\newcommand{\note}[2][]{\added[#1,remark={#2}]{}}   
%---layout----- margins
\usepackage[top=2.5cm,bottom=2.0cm,left=3.0cm,right=2.0cm]{geometry}
%---language-----
\usepackage[english,brazilian]{babel}
\usepackage[utf8]{inputenc}
\usepackage[T1]{fontenc}
%--- type font-----
\usepackage{times}
%--title formats -- layouts
\usepackage[indentafter]{titlesec}
  \setlength{\parindent}{2.5cm}
% -- special numbering
\usepackage{fancyhdr}
\fancypagestyle{plain}{ ---
  \fancyhf{}
  \renewcommand{\headrulewidth}{0pt}
  \fancyhead[R]{\thepage}
}  
%--- level
\setcounter{secnumdepth}{5}
%---math options
\usepackage{amsfonts}
\usepackage{amsmath,amssymb,textcomp}
%---graphical options
\usepackage[]{graphicx}
%--table of contents options
\usepackage{tocloft}
  \renewcommand\cftsecafterpnum{\vskip14pt}
  \renewcommand\cftsubsecafterpnum{\vskip14pt}
  \usepackage[nottoc,notlof,notlot]{tocbibind} %print reference in toc 
  \usepackage[titletoc,toc,page]{appendix}
   \renewcommand{\appendixname}{Apêndice}
   \renewcommand{\appendixtocname}{Apêndice}
   \renewcommand{\appendixpagename}{Apêndice}
%---hyperlinks
  \usepackage{hyperref}
   \hypersetup{colorlinks=true, linkcolor=blue, citecolor=blue, filecolor=blue, urlcolor=green,
              pdfauthor={Author name},
              pdftitle={title-pdf},
              pdfsubject={Theses},
              pdfkeywords={Keyword1, Keyword2},
              pdfproducer={Latex},
              pdfcreator={pdflatex}}%              
%-------------------------------------------------------
\usepackage[portuguese]{nomencl}
\renewcommand{\nomname}{Lista de Abreviaturas}
 \makenomenclature
%-------------------------------------------------------
%----- title pages -------
% name sections adapted to portuguese

% Master document %--Thesis
%--Author: Lozada-Blanco Asdrubal
%--the following is cc by-sa 3.0 with attribution required
% any request to e-mail: aslozada@gmail.com 
%---------------------------------------------------------------------------------------------------------
%--UNOFFICIAL template to master's and doctoral thesis in concordance with instruction from PPGQ-UFSCar
% https://www.ppgq.ufscar.br/regimentos-normas/normas-redacao-dissertacao-e-tese
%----------------------------------------------------------------------------------------------------------

% pré-textuais

\newcommand{\instituto}{UNIVERSIDADE FEDERAL DE SÃO CARLOS}
\newcommand{\centro}{CENTRO DE CIÊNCIAS EXATAS E DE TECNOLOGIA}
\newcommand{\programa}{PROGRAMA DE PÓS-GRADUAÇÃO EM QUÍMICA}
\newcommand{\departamento}{DEPARTAMENTO DE QUÍMICA}
\newcommand{\nombre}{Nome Sobrenome}
\newcommand{\orientador}{Prof. Dr. Nome sobrenome}
\newcommand{\titulo}{\textbf{}}
\newcommand{\cidade}{SÃO CARLOS - SP}
\newcommand{\data}{2022}
\newcommand{\bolsa}{Bolsista Ag\^encia de fomento}

% ---
\newcommand{\folhaR}{
     \pagestyle{empty}
     \begin{titlepage} 
         \begin{center}
           {\Large \instituto}\\
           {\Large \centro}\\
           {\Large \departamento}\\
           {\Large \programa}\\
          \par
           \vspace{250pt}
           {\large \titulo}
          \par
           \vspace{85pt}
           \hspace*{250pt}\parbox{9.0cm}{{\Large \nombre}}
           \vfill
           {\large \cidade}\\
           {\large \data}
         \end{center}
       \end{titlepage}
  }
% --
  \newcommand{\folhaRR}{
    \pagestyle{empty}
    \begin{titlepage}
     \begin{center}
      {\Large \instituto}\\
      {\Large \centro}\\
      {\Large \departamento}\\
      {\Large \programa}\\
      \par
      \vspace{180pt}
      {\large \titulo}
      \par
      \vspace{60pt}
      \hspace*{250pt}\parbox{9.0cm}{{\Large \nombre${\dagger}$}}
      \par
      \vspace{60pt}
      \hspace*{150pt}\parbox{9.0cm}{{\large Tese apresentada como parte dos requisitos para obtenção do título de DOUTOR EM CIÊNCIAS, área de concentração: FÍSICO-QUÍMICA}} 
      \par
      \vspace{60pt}
      \hspace*{0pt}\parbox{20.0cm}{{\large Orientador: \orientador}}
      \par
      \vspace{20pt}
      \hspace*{0pt}\parbox{20.0cm}{{\large $\dagger$ \bolsa}}
      \par
      \vfill
      {\large \cidade}\\
      {\large \data}
     \end{center}
     \end{titlepage}
  }
 %--
 \newcommand{\aprovacao}{
   \newpage
   \input{aprovacao.tex} %dummy
 }
 %---
 \newcommand{\catalografica}{
  \newpage
  %dummy page
\begin{center}
\textcolor{gray}{\Huge{Intentionally blank page}}
\end{center}
 %dummy
 }
 %--
 \newcommand{\dedicatoria}{
   \newpage
    \vspace*{0.75\textheight}
     \begin{flushright}
      \emph{Ad nihil}
     \end{flushright}     
 }
%--
 \newcommand{\agradecimento}{
   \newpage
    \noindent{\Large\textbf{Agradecimentos}}
    \vskip 2.0cm
    \noindent Lorem ipsum dolor sit amet, consetetur sadipscing elitr, sed diam nonumy eirmod tempor invidunt ut labore et dolore magna aliquyam erat, sed diam voluptua. At vero eos et accusam et justo duo dolores et ea rebum. Stet clita kasd gubergren, no sea takimata sanctus est Lorem ipsum dolor sit amet. Lorem ipsum dolor sit amet, consetetur sadipscing elitr, sed diam nonumy eirmod tempor invidunt ut labore et dolore magna aliquyam erat, sed diam voluptua. At vero eos et accusam et justo duo dolores et ea rebum. Stet clita kasd gubergren, no sea takimata sanctus est Lorem ipsum dolor sit amet. \\
    
    \noindent Lorem ipsum dolor sit amet, consetetur sadipscing elitr, sed diam nonumy eirmod tempor invidunt ut labore et dolore magna aliquyam erat, sed diam voluptua. At vero eos et accusam et justo duo dolores et ea rebum. Stet clita kasd gubergren, no sea takimata sanctus est Lorem ipsum dolor sit amet. Lorem ipsum dolor sit amet, consetetur sadipscing elitr, sed diam nonumy eirmod tempor invidunt ut labore et dolore magna aliquyam erat, sed diam voluptua. At vero eos et accusam et justo duo dolores et ea rebum. Stet clita kasd gubergren, no sea takimata sanctus est Lorem ipsum dolor sit amet. \\
 }
%--
 \newcommand{\abreviatura}{
   \newpage
   \pagestyle{myheadings}
    \pagenumbering{roman}
    \setcounter{page}{7}

     %nomenclature
\nomenclature{MC}{Monte Carlo}
\nomenclature{FORTRAN}{Formula Translator}
\nomenclature{UFSCar}{Universidade Federal de São Carlos}



     \printnomenclature
 }
%--
\newcommand{\tabelas}{
  \newpage
  \pagestyle{plain}
   \pagenumbering{roman}
   \setcounter{page}{8}
   \listoftables}
%--
\newcommand{\figuras}{
  \newpage
  \listoffigures
}
%--   
 \newcommand{\resumo}{
   \newpage
   \pagestyle{myheadings}
   \pagenumbering{roman}
   \setcounter{page}{10}
   \markboth{}{}
   \vspace*{10pt}
   \begin{center}
    \emph{\begin{large}Resumo\end{large}}\label{resumo}
    \vspace{2pt}
  \end{center} 
  % Resumo
\noindent El presente volumen, \textit{Procesos estocásticos}, es una traducción española de mi libro escrito en japonés y editado por Iwanami shoten en 1957 en dos partes: Procesos estocásticos I y II. En este trabajo, brindo información unificada y comprensible de procesos aditivos, procesos estacionarios, y procesos de MARKOV, que en la actualidad son la clase de procesos estocásticos más importantes.\\\\ 
\noindent El presente volumen, \textit{Procesos estocásticos}, es una traducción española de mi libro escrito en japonés y editado por Iwanami shoten en 1957 en dos partes: Procesos estocásticos I y II. En este trabajo, brindo información unificada y comprensible de procesos aditivos, procesos estacionarios, y procesos de MARKOV, que en la actualidad son la clase de procesos estocásticos más importantes.\\\\ 
\noindent El presente volumen, \textit{Procesos estocásticos}, es una traducción española de mi libro escrito en japonés y editado por Iwanami shoten en 1957 en dos partes: Procesos estocásticos I y II. En este trabajo, brindo información unificada y comprensible de procesos aditivos, procesos estacionarios, y procesos de MARKOV, que en la actualidad son la clase de procesos estocásticos más importantes.\\\\ 
\noindent El presente volumen, \textit{Procesos estocásticos}, es una traducción española de mi libro escrito en japonés y editado por Iwanami shoten en 1957 en dos partes: Procesos estocásticos I y II. En este trabajo, brindo información unificada y comprensible de procesos aditivos, procesos estacionarios, y procesos de MARKOV, que en la actualidad son la clase de procesos estocásticos más importantes.\\\\ 
\noindent El presente volumen, \textit{Procesos estocásticos}, es una traducción española de mi libro escrito en japonés y editado por Iwanami shoten en 1957 en dos partes: Procesos estocásticos I y II. En este trabajo, brindo información unificada y comprensible de procesos aditivos, procesos estacionarios, y procesos de MARKOV, que en la actualidad son la clase de procesos estocásticos más importantes.\\\\ 
\par
\vspace{1em}
\noindent\textbf{Palavras chave:} Markov, Wentzell, Yale

}
  %--
\newcommand{\abstracts}{
  \newpage
   \vspace*{10pt}
   \begin{center}
    \emph{\begin{large}Abstract\end{large}}\label{abstract}
    \vspace{2pt}
  \end{center} 
  %abstract
\noindent The present volume, \textit{Essentials of Stochastic Process}, is
an English translation of my book written in Japanese and issued by Iwanami
Shoten in 1957 in two parts: Stochastic Processes I and II. In this  work, I
provide a unified and comprehensive account of additive processes, stationary processes, and Markov processes, which remain to this day the three most important classes of stochastic processes.\\\\
\noindent The present volume, \textit{Essentials of Stochastic Process}, is
an English translation of my book written in Japanese and issued by Iwanami
Shoten in 1957 in two parts: Stochastic Processes I and II. In this  work, I
provide a unified and comprehensive account of additive processes, stationary processes, and Markov processes, which remain to this day the three most important classes of stochastic processes.\\\\
\noindent The present volume, \textit{Essentials of Stochastic Process}, is
an English translation of my book written in Japanese and issued by Iwanami
Shoten in 1957 in two parts: Stochastic Processes I and II. In this  work, I
provide a unified and comprehensive account of additive processes, stationary processes, and Markov processes, which remain to this day the three most important classes of stochastic processes.\\\\
\noindent The present volume, \textit{Essentials of Stochastic Process}, is
an English translation of my book written in Japanese and issued by Iwanami
Shoten in 1957 in two parts: Stochastic Processes I and II. In this  work, I
provide a unified and comprehensive account of additive processes, stationary processes, and Markov processes, which remain to this day the three most important classes of stochastic processes.\\\\
\noindent The present volume, \textit{Essentials of Stochastic Process}, is
an English translation of my book written in Japanese and issued by Iwanami
Shoten in 1957 in two parts: Stochastic Processes I and II. In this  work, I
provide a unified and comprehensive account of additive processes, stationary processes, and Markov processes, which remain to this day the three most important classes of stochastic processes.\\\\
\noindent The present volume, \textit{Essentials of Stochastic Process}, is
an English translation of my book written in Japanese and issued by Iwanami
Shoten in 1957 in two parts: Stochastic Processes I and II. In this  work, I
provide a unified and comprehensive account of additive processes, stationary processes, and Markov processes, which remain to this day the three most important classes of stochastic processes.
\par
\vspace{1em}
\noindent\textbf{Keywords:} Markov, Wentzell, Yale


   }
 %--
 \newcommand{\sumario}{
  \newpage
  \pagestyle{plain}
   \pagenumbering{roman}
   \setcounter{page}{12}
   \tableofcontents
 }


  %pré-textuais

%--- Master document ----  
\begin{document}

% seção de pretextuais de acordo com a norma da PPGQ

\folhaR
\folhaRR
\aprovacao
\catalografica
\dedicatoria
\agradecimento
\abreviatura
\tabelas
\figuras
\resumo
\abstracts %the word "abstract" is reserved
\sumario
% fin da seção pré-textuais  

%--thesis--
 
  %introducao


\newpage    
 \pagestyle{myheadings}  
  \pagenumbering{arabic}
    \setcounter{page}{1}
     \markboth{}{}
    
 \section{Introdução}
 Muchos de los problemas dificiles e interesantes en mecánica estadística aparecen
 cuando las partículas constituyentes de un sistema interacutan una con otra con
 energias pares o múltiples. Los tipos de comportamiento que ocurren en los sistemas
 debidos a estas interacciones son conocidos como \textit{fenómenos coperativos} dado que aparecen en muchos casos de transición de fase. este libro y su volumen complementario están principalmente relacionados con transiciones de fase de sistemas 
 reticulares. Debido principalmente al significado ganado desde la teoria de
 escalamiento y los métodos de renormalización de grupos, estos temas han sido 
 desarrollados rápidamente en los últimos treinta años.\\

 Una división amplia del material puede ser hecha entre resultados exactos y métodos aproximados. Hemos encontrado apropiado incluir algunas discusiones de resultados 
 exactos en este y el segundo volumen. El otro área de discusión en este volumen
 es la teoria de campo médio. Para sistemas complicadis algunos tipos de métods de campo medio son a menudo tratables.\\

 El trabajo en estos dos volumenes fué desarrollado en parte en los cursos dictados
 por George Bell y yo en el Departamento de matemáticas de la Universidad de Manchester. Preguntas y comentarios de los estudiantes en estos cursos han ayudado a reducir errores y algunas cosas incomprendidas.\\
     
 Muchos de los problemas dificiles e interesantes en mecánica estadística aparecen
 cuando las partículas constituyentes de un sistema interacutan una con otra con
 energias pares o múltiples. Los tipos de comportamiento que ocurren en los sistemas
 debidos a estas interacciones son conocidos como \textit{fenómenos coperativos} dado que aparecen en muchos casos de transición de fase. este libro y su volumen complementario están principalmente relacionados con transiciones de fase de sistemas 
 reticulares. Debido principalmente al significado ganado desde la teoria de
 escalamiento y los métodos de renormalización de grupos, estos temas han sido 
 desarrollados rápidamente en los últimos treinta años.\\

 Una división amplia del material puede ser hecha entre resultados exactos y métodos aproximados. Hemos encontrado apropiado incluir algunas discusiones de resultados 
 exactos en este y el segundo volumen. El otro área de discusión en este volumen
 es la teoria de campo médio. Para sistemas complicadis algunos tipos de métods de campo medio son a menudo tratables.\\

 El trabajo en estos dos volumenes fué desarrollado en parte en los cursos dictados
 por George Bell y yo en el Departamento de matemáticas de la Universidad de Manchester. Preguntas y comentarios de los estudiantes en estos cursos han ayudado a reducir errores y algunas cosas incomprendidas.\\
  
 El trabajo en estos dos volumenes fué desarrollado en parte en los cursos dictados
 por George Bell y yo en el Departamento de matemáticas de la Universidad de Manchester. Preguntas y comentarios de los estudiantes en estos cursos han ayudado a reducir errores y algunas cosas incomprendidas.\\
  
 El trabajo en estos dos volumenes fué desarrollado en parte en los cursos dictados
 por George Bell y yo en el Departamento de matemáticas de la Universidad de Manchester. Preguntas y comentarios de los estudiantes en estos cursos han ayudado a reducir errores y algunas cosas incomprendidas.\\
  

 El trabajo en estos dos volumenes fué desarrollado en parte en los cursos dictados
 por George Bell y yo en el Departamento de matemáticas de la Universidad de Manchester. Preguntas y comentarios de los estudiantes en estos cursos han ayudado a reducir errores y algunas cosas incomprendidas.\\
  

 El trabajo en estos dos volumenes fué desarrollado en parte en los cursos dictados
 por George Bell y yo en el Departamento de matemáticas de la Universidad de Manchester. Preguntas y comentarios de los estudiantes en estos cursos han ayudado a reducir errores y algunas cosas incomprendidas.\\
  

 El trabajo en estos dos volumenes fué desarrollado en parte en los cursos dictados
 por George Bell y yo en el Departamento de matemáticas de la Universidad de Manchester. Preguntas y comentarios de los estudiantes en estos cursos han ayudado a reducir errores y algunas cosas incomprendidas.\\
  
\newpage 

  %teoria

\newpage 
 \section{Teoria}
  \subsection{Mec\^{a}nica qu\^{a}ntica}
   Lorem ipsum dolor sit amet, consetetur sadipscing elitr, sed diam nonumy eirmod tempor invidunt ut labore et dolore magna aliquyam erat, sed diam voluptua. At vero eos et accusam et justo duo dolores et ea rebum. Stet clita kasd gubergren, no sea takimata sanctus est Lorem ipsum dolor sit amet. Lorem ipsum dolor sit amet, consetetur sadipscing elitr, sed diam nonumy eirmod tempor invidunt ut labore et dolore magna aliquyam erat, sed diam voluptua. At vero eos et accusam et justo duo dolores et ea rebum. Stet clita kasd gubergren, no sea takimata sanctus est Lorem ipsum dolor sit amet. Lorem ipsum dolor sit amet, consetetur sadipscing elitr, sed diam nonumy eirmod tempor invidunt ut labore et dolore magna aliquyam erat, sed diam voluptua. At vero eos et accusam et justo duo dolores et ea rebum. Stet clita kasd gubergren, no sea takimata sanctus est Lorem ipsum dolor sit amet.   
   
   Duis autem vel eum iriure dolor in hendrerit in vulputate velit esse molestie consequat, vel illum dolore eu feugiat nulla facilisis at vero eros et accumsan et iusto odio dignissim qui blandit praesent luptatum zzril delenit augue duis dolore te feugait nulla facilisi. Lorem ipsum dolor sit amet, consectetuer adipiscing elit, sed diam nonummy nibh euismod tincidunt ut laoreet dolore magna aliquam erat volutpat.   
   
   Ut wisi enim ad minim veniam, quis nostrud exerci tation ullamcorper suscipit lobortis nisl ut aliquip ex ea commodo consequat. Duis autem vel eum iriure dolor in hendrerit in vulputate velit esse molestie consequat, vel illum dolore eu feugiat nulla facilisis at vero eros et accumsan et iusto odio dignissim qui blandit praesent luptatum zzril delenit augue duis dolore te feugait nulla facilisi.   
   
   Nam liber tempor cum soluta nobis eleifend option congue nihil imperdiet doming id quod mazim placerat facer possim assum. Lorem ipsum dolor sit amet, consectetuer adipiscing elit, sed diam nonummy nibh euismod tincidunt ut laoreet dolore magna aliquam erat volutpat. Ut wisi enim ad minim veniam, quis nostrud exerci tation ullamcorper suscipit lobortis nisl ut aliquip ex ea commodo consequat.   
   
   Duis autem vel eum iriure dolor in hendrerit in vulputate velit esse molestie consequat, vel illum dolore eu feugiat nulla facilisis.   
   
   At vero eos et accusam et justo duo dolores et ea rebum. Stet clita kasd gubergren, no sea takimata sanctus est Lorem ipsum dolor sit amet. Lorem ipsum dolor sit amet, consetetur sadipscing elitr, sed diam nonumy eirmod tempor invidunt ut labore et dolore magna aliquyam erat, sed diam voluptua. At vero eos et accusam et justo duo dolores et ea rebum. Stet clita kasd gubergren, no sea takimata sanctus est Lorem ipsum dolor sit amet. Lorem ipsum dolor sit amet, consetetur sadipscing elitr, At accusam aliquyam diam diam dolore dolores duo eirmod eos erat, et nonumy sed tempor et et invidunt justo labore Stet clita ea et gubergren, kasd magna no rebum. sanctus sea sed takimata ut vero voluptua. est Lorem ipsum dolor sit amet. Lorem ipsum dolor sit amet, consetetur\\

   sanctus sea sed takimata ut vero voluptua.  $\psi(x,t)$, [...] At accusam aliquyam diam diam dolore dolores duo eirmod eos erat, et nonumy sed tempor et et invidunt justo labore Stet clita ea et gubergren, kasd magna no rebum. sanctus sea sed takimata ut vero voluptua.
   
   \begin{equation}
     \psi(x,t)=\frac{1}{\sqrt{2\pi}}\int \hat{\psi}(k,0) e^{i(kx-\frac{\hbar k^2 t}{2m})}dk
   \end{equation}

   \subsection{Sistemas magnéticos}
    Nam liber tempor cum soluta nobis eleifend option congue nihil imperdiet doming id quod mazim placerat facer possim assum. Lorem ipsum dolor sit amet, consectetuer adipiscing elit, sed diam nonummy nibh euismod tincidunt ut laoreet dolore magna aliquam erat volutpat. Ut wisi enim ad minim veniam, quis nostrud exerci tation ullamcorper suscipit lobortis nisl ut aliquip ex ea commodo consequat. 
    
    \begin{equation}
      \chi_{T}=\left(\frac{\partial \mathcal{M}}{\partial \mathcal{H}}\right)
    \end{equation}
    \subsection{Equilibrio de fase}
    Nam liber tempor cum soluta nobis eleifend option congue nihil imperdiet doming id quod mazim placerat facer possim assum. Lorem ipsum dolor sit amet, consectetuer adipiscing elit, sed diam nonummy nibh euismod tincidunt ut laoreet dolore magna aliquam erat volutpat. Ut wisi enim ad minim veniam, quis nostrud exerci tation ullamcorper suscipit lobortis nisl ut aliquip ex ea commodo consequat.  figura~\ref{fig:julia}
    \begin{figure}[!h]
      \centering
      \includegraphics[scale=0.5]{julia}
      \caption{Julia set}
      \label{fig:pollito}
    \end{figure}  
    \subsubsection{Din\^amica estoc\'astica}
       At vero eos et accusam et justo duo dolores et ea rebum. Stet clita kasd gubergren, no sea takimata sanctus est Lorem ipsum dolor sit amet. Lorem ipsum dolor sit amet, consetetur sadipscing elitr, sed diam nonumy eirmod tempor invidunt ut labore et dolore magna aliquyam erat, sed diam voluptua. At vero eos et accusam et justo duo dolores et ea rebum. Stet clita kasd gubergren, no sea takimata sanctus est Lorem ipsum dolor sit amet. Lorem ipsum dolor sit amet, consetetur sadipscing elitr, At accusam aliquyam diam diam dolore dolores duo eirmod eos erat, et nonumy sed tempor et et invidunt justo labore Stet clita ea et gubergren, kasd magna no rebum. sanctus sea sed takimata ut vero voluptua. est Lorem ipsum dolor sit amet. Lorem ipsum dolor sit amet, consetetur\\
    
    sanctus~\cite{ALE08} sea sed takimata ut vero voluptua.  $\psi(x,t)$, [...] At accusam aliquyam diam diam dolore dolores duo eirmod eos erat, et nonumy sed tempor et et invidunt justo labore Stet clita ea et gubergren, kasd magna no rebum. sanctus sea sed takimata ut vero voluptua. Tabela ~\ref{tab:a}
    \begin{table}[!htb]
        \begin{tabular}{lrc}\hline
          name & mark & grade \\
          \hline
          bla & bla & bla \\
          bla & bla & bla \\
          bla & bla & bla \\ \hline
        \end{tabular}
        \caption{Tabela}\label{tab:a}
    \end{table}
    \newpage
\newpage

  %metologia
\newpage 
 \section{Metodologia}
  \subsection{El programa}
   Suponemos que el lector tiene conocimiento prévio de tremodinámica.
   Nuestra intención aquí es revisar brevemente la parte esencial de la
   teoria, para introducir a fenómenos críticos y, al final del capítulo
   poner la teoria en una forma más generalizada que en el inicio.\\

   Así, para tal cambio,..., los pares de variables $(S^{~},T^{~})$, ...,
   ocurren,... de esto sigue que
   \begin{equation}
     T^{~}=\left(\frac{\partial U}{\partial S^{~}}\right)
   \end{equation}

   \subsection{el código}
    Otra funcion de respuesta importante para sistemas magnéticos es la 
    \textit{suceptibilidad magnética isotérmica}$\chi_{T}$ que está definida
    por unidad de volumen de materia.
    \begin{equation}
      \chi_{T}=\left(\frac{\partial \mathcal{M}}{\partial \mathcal{H}}\right)
    \end{equation}
    \subsection{Equilibrio de fase}
    Si las curvas de temperatura constante son trazadas en el plano, se encunetra que, ..., mostrado en la figura~\ref{fig:pollitos}
    \begin{figure}[!h]
      \centering
      \includegraphics[scale=0.5]{frango}
      \caption{esto otro pollito}
      \label{fig:pollitos}
    \end{figure}  
    \subsubsection{el segmento}
    Retornando al sistema general, la densidad de energia libre $f_{\eta}$ definida por la ecuación. Como es mostrado en la tabla~\ref{tab:b}
    \begin{table}[!htb]
        \begin{tabular}{lrc}\hline
          name & mark & grade \\
          \hline
          bla & bla & bla \\
          bla & bla & bla \\
          bla & bla & bla \\ \hline
        \end{tabular}
        \caption{esto es otra tabla}\label{tab:b}
    \end{table}
    \newpage


  %resultados
\newpage 
 \section{resultados}
 \subsection{del código}
  alguna cosa vá aqui
 \subsection{de los algoritmos}
 aquí también
 \newpage


  %apendices
\newpage

 \appendix
 \newpage
 \section{Apêndice}
 \subsection{Primeiro apêndice}
 \subsection{segundo apêndice}
 \section{Apêndice}
 \subsection{Primeiro apêndice}
 \subsection{segundo apêndice}
 \section{Apêndice}
 \subsection{Primeiro apêndice}
 \subsection{segundo apêndice}
\newpage

  
  % add the full path to bibliographic style and references
  \bibliographystyle{ppgq}
  \bibliography{referencias}
%
\end{document}
