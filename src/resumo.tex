% Resumo
\noindent El presente volumen, \textit{Procesos estocásticos}, es una traducción española de mi libro escrito en japonés y editado por Iwanami shoten en 1957 en dos partes: Procesos estocásticos I y II. En este trabajo, brindo información unificada y comprensible de procesos aditivos, procesos estacionarios, y procesos de MARKOV, que en la actualidad son la clase de procesos estocásticos más importantes.\\\\ 
\noindent El presente volumen, \textit{Procesos estocásticos}, es una traducción española de mi libro escrito en japonés y editado por Iwanami shoten en 1957 en dos partes: Procesos estocásticos I y II. En este trabajo, brindo información unificada y comprensible de procesos aditivos, procesos estacionarios, y procesos de MARKOV, que en la actualidad son la clase de procesos estocásticos más importantes.\\\\ 
\noindent El presente volumen, \textit{Procesos estocásticos}, es una traducción española de mi libro escrito en japonés y editado por Iwanami shoten en 1957 en dos partes: Procesos estocásticos I y II. En este trabajo, brindo información unificada y comprensible de procesos aditivos, procesos estacionarios, y procesos de MARKOV, que en la actualidad son la clase de procesos estocásticos más importantes.\\\\ 
\noindent El presente volumen, \textit{Procesos estocásticos}, es una traducción española de mi libro escrito en japonés y editado por Iwanami shoten en 1957 en dos partes: Procesos estocásticos I y II. En este trabajo, brindo información unificada y comprensible de procesos aditivos, procesos estacionarios, y procesos de MARKOV, que en la actualidad son la clase de procesos estocásticos más importantes.\\\\ 
\noindent El presente volumen, \textit{Procesos estocásticos}, es una traducción española de mi libro escrito en japonés y editado por Iwanami shoten en 1957 en dos partes: Procesos estocásticos I y II. En este trabajo, brindo información unificada y comprensible de procesos aditivos, procesos estacionarios, y procesos de MARKOV, que en la actualidad son la clase de procesos estocásticos más importantes.\\\\ 
\par
\vspace{1em}
\noindent\textbf{Palavras chave:} Markov, Wentzell, Yale
